Le langage jouet doit permettre à l'utilisateur de produire un exemple de code simple reprenant les principales structures (boucles, branchements conditionnels,...)

\begin{lstfloat}[h!]
	\centering
\inputPython{example2.code}{1}{11}
\caption{Exemple de code dans le langage jouet}
\end{lstfloat}

\subsubsection{Expressions admissibles}

Les expression admissibles sont présentées dans la table  \ref{tab:expressions} ci-dessous.
\begin{table}[h!]\caption{\label{tab:expressions}Expressions admissibles}
	\centering
	\begin{tabular}{|c|c|c|}
	\hline
	\multicolumn{2}{|c|}{Variable}   &            \pyinline{x}            \\ \hline
	\multicolumn{2}{|c|}{Entier}     &            \pyinline{n}            \\ \hline
	              &      Somme       &         \pyinline{e1 + e2}         \\ \cline{2-3}
	              &    Différence    &         \pyinline{e1 - e2}         \\ \cline{2-3}
	 Opérations   &     Produit      &         \pyinline{e1 * e2}         \\ \cline{2-3}
	arithmétiques & Division entière &         \pyinline{e1 / e2}         \\ \cline{2-3}
	              &      Reste       & \pyinline{e1 % e2
	              } \\ \cline{2-3}
	              &      Opposé      &           \pyinline{-e1}           \\ \hline
\end{tabular}
\hspace{3cm}
\begin{tabular}{|c|c|c|}
	\hline
	\multicolumn{3}{|c|}{Opérations logiques}                                         \\ \hline
	\multirow{10}{*}{Binaires} &           Egalité           &  \pyinline{e1 == e2}   \\ \cline{2-3}
	                           &         Différence          &  \pyinline{e1 != e2}   \\ \cline{2-3}
	                           & \multirow{4}{*}{Inégalités} &   \pyinline{e1 < e2}   \\ \cline{3-3}
	                           &                             &   \pyinline{e1 > e2}   \\ \cline{3-3}
	                           &                             &  \pyinline{e1 <= e2}   \\ \cline{3-3}
	                           &                             &  \pyinline{e1 >= e2}   \\ \cline{2-3}
	                           &     \multirow{2}{*}{Et}     &  \pyinline{e1 and e2}  \\ \cline{3-3}
	                           &                             &  \pyinline{e1 \& e2}   \\ \cline{2-3}
	                           &     \multirow{2}{*}{Ou}     &  \pyinline{e1 or e2}   \\ \cline{3-3}
	                           &                             & \pyinline{e1     | e2} \\ \hline
	 \multirow{2}{*}{Unaire}   &      inverse bit à bit      &     \pyinline{~e1}     \\ \cline{2-3}
	                           &      négation logique       &   \pyinline{not e1}    \\ \hline
\end{tabular}
\end{table}
%
\subsubsection{Liste de commandes admissibles}

\begin{minipage}[c]{0.38\textwidth}
	\begin{minted}[frame = single, label={Affectation}]{python}
x=e
\end{minted}
\end{minipage}
\hspace{1cm}
\begin{minipage}[c]{0.38\textwidth}
	avec \pyinline{e} une expression logique ou arithmétique.
\end{minipage}
\vspace{0.5cm}

\begin{minipage}[c]{0.38\textwidth}
\begin{minted}[frame = single, label={Branchement conditionnel}]{python}
if e :
	c1
elif e2: 
	c2
else:
	c3
\end{minted}
\end{minipage}
\hspace{1cm}
\begin{minipage}[c]{0.38\textwidth}
	avec \pyinline{e1} et \pyinline{e2} des expressions et \pyinline{c1}, \pyinline{c2} et \pyinline{c3} des commandes.
	
	Les branchement \pyinline{else} et \pyinline{elif} sont optionnels.
\end{minipage}
\vspace{0.5cm}

\begin{minipage}[c]{0.38\textwidth}
\begin{minted}[frame = single, label={Boucle}]{python}
while e :
	c1
\end{minted}
\end{minipage}
\hspace{1cm}
\begin{minipage}[c]{0.38\textwidth}
	avec \pyinline{e} une expression et \pyinline{c} une commande.	
\end{minipage}
\vspace{0.5cm}

\begin{minipage}[c]{0.38\textwidth}
\begin{minted}[frame = single, label={Lecture clavier}]{python}
input()
\end{minted}
\end{minipage}
\vspace{0.5cm}

\begin{minipage}[c]{0.38\textwidth}
\begin{minted}[frame = single, label={Ecriture sur la sortie courante}]{python}
print(v)
\end{minted}
\end{minipage}
\hfill
\begin{minipage}[c]{0.38\textwidth}
	avec \pyinline{e} une expression	
\end{minipage}

\subsubsection{Indentations}
Le code est indenté comme en python afin de détecter les blocs:
\begin{itemize}
	\item L'indentation n'augmente qu'après un \pyinline{:} lié à une structure \pyinline{if} ou \pyinline{while}
	\item L'indentation ne peut diminuer que atteindre un niveau précédemment atteint.
\end{itemize}

\subsubsection{Commentaires}
Les commentaires sont repérés par le caractère \pyinline{#} .

\begin{lstfloat}
	\begin{center}
	\begin{pythonlst}
aaaaaaaaaa:
	aaaaaaaaaa
	aaaaaaaaaa
		aaaaaaaaaa # pas valable il manque : avant
		aaaaaaaaaa
	aaaaaaaaaa
aaaaaaaaaa
aaaaaaaaaa:
			aaaaaaaaaa
	aaaaaaaaaa # pas valable. Ce niveau a été atteint avant mais pas dans le même bloc
aaaaaaaaaa
\end{pythonlst}
\caption{Langage jouet - Commentaires et indentations}
\end{center}
\end{lstfloat}



